\chapter{Dispersive Material \label{chap:dispersive}}

%\setcounter{page}{1}

\renewcommand{\thefootnote}{\fnsymbol{footnote}}
\footnotetext[2]{Lecture notes by John Schneider.  {\tt
fdtd-dispersive-material.tex}}

\section{Introduction}

For many problems one can obtain acceptably accurate results
by assuming material parameters are constants.  However, constant
material parameters are inherently an approximation.  For example, it
is impossible to have a lossless dielectric with constant permittivity
(except, of course, for free space).  If such a material did exist it
would violate causality.  (For a material to behave causally, the
Kramers-Kronig relations show that for any deviation from free-space
behavior the imaginary part of the permittivity or permeability, i.e.,
the loss, cannot vanish for all frequencies.  Nevertheless, as far as
causality is concerned, the loss can be arbitrarily small.)

A non-unity, scalar, constant relative permittivity is equivalent to
assuming the polarization of charge within a material is instantaneous
and in perfect proportion to the applied electric field.  Furthermore,
the reaction is the same at all frequencies, is the same in all
directions, is the same for all times, and the same proportionality
constant holds for all field strengths.  In reality, essentially none
of the these assumptions are absolutely correct.  The relationship
between the electric flux density $\Dvec$ and the electric field
$\Evec$ can reflect all the complexity of the real world.  Instead of
simply having $\Dvec=\epsilon\Evec$ where $\epsilon$ is a scalar
constant, one can make $\epsilon$ a tensor to describe different
behaviors in different directions (off diagonal terms would indicate
the amount of coupling from one direction to another).  The
permittivity can also be written as a nonlinear function of the
applied electric field to account for nonlinear media.  The material
parameters can be functions of time (such as might pertain to a
material which is being heated).  Finally, one should not forget that
the permittivity can be a function of position to account for spatial
inhomogeneities.  

When the speed of light in a material is a function of frequency, the
material is said to be dispersive.  The fact that the FDTD grid is
dispersive has been discussed in Chap.\ \ref{chap:dispersion}.  That
dispersion is a numerical artifact and is distinct from the 
subject of this chapter.  We have also considered lossy materials.
Even when the conductivity of a material is assumed to be constant,
the material is dispersive (ref.\ \refeq{eq:lossyK} which shows that
the phase constant is not linearly proportional to the frequency which
must be the case for non-dispersive propagation).

When the permittivity or permeability of a material are functions of
frequency, the material is dispersive.  In time-harmonic form one can
account for the frequency dependence of permittivity by writing
$\hDvec(\omega)=\hat{\epsilon}(\omega)\hEvec(\omega)$, where a caret
is used to indicate a quantity in the frequency domain.  This
expression is simple in the frequency domain, but the FDTD method is a
time-domain technique.  The multiplication of harmonic functions is
equivalent to convolution in the time domain.  Therefore it requires
some additional effort to model these types of dispersive materials.
We start with a brief review of dispersive materials and then consider
two ways in which to model such materials in the FDTD method.

\section{Constitutive Relations and Dispersive Media}

The electric flux density and magnetic field are related to the
electric field and the magnetic flux density via
\begin{eqnarray}
  \Dvec &=& \epsilon_0 \Evec + \mathbf{P}, \label{eq:dAndEAndP}\\
  \Hvec &=& \frac{\Bvec}{\mu_0} - \mathbf{M}, \label{eq:hAndBAndM}
\end{eqnarray}
where $\mathbf{P}$ and $\mathbf{M}$ account for the electric and
magnetic dipoles, respectively, induced in the media.  Keep in mind
that force on a charge is a function of $\Evec$ and $\Bvec$ so in some
sense $\Evec$ and $\Bvec$ are the ``real'' fields.  The polarization
vector $\mathbf{P}$ accounts for the local displacement of bound
charge in a material.  Because of the way in which $\mathbf{P}$ is
constructed, by adding it to $\epsilon_0\Evec$ the resulting electric
flux density $\Dvec$ has the local effect of bound charge removed.  In
this way, the integral of $\Dvec$ over a closed surface yields the
free charge (thus Gauss's law, as expressed using the $\Dvec$ field,
is true whether material is present or not).

The magnetic field $\Hvec$ ignores the local effect of bound charge in
motion.  Thus, the integration of $\Hvec$ over a closed loop yields
the current flowing through the surface enclosed by that loop where
the current is due to either the flow of free charge or displacement
current (i.e., the integral form of Ampere's law).  Rearranging the
terms in \refeq{eq:hAndBAndM} and multiplying by $\mu_0$ yields an
expression for the magnetic flux density\footnote{Note that there are
  those (e.g., Feynman) who advocate that one should avoid using
  $\Dvec$ and $\Hvec$.  Others (e.g., Sommerfeld) have discussed the
  unfortunate naming of the magnetic field and the magnetic flux
  density.  However, those issues are peripheral to the main subject
  of interest here and we will employ the notation and usage as is
  commonly found in engineering electromagnetics.}, i.e.,
\begin{equation}
  \Bvec = \mu_0(\Hvec + \mathbf{M}).
\end{equation}

At a given frequency, for a linear, isotropic medium, the polarization
vectors can be related to the electric and magnetic fields via an
electric or magnetic susceptibility
\begin{eqnarray}
  \hPvec(\omega) &=&
        \epsilon_0\hat{\chi}_e(\omega)\hEvec(\omega), \\
  \hMvec(\omega) &=&
        \hat{\chi}_m(\omega)\hHvec(\omega),
\end{eqnarray}
where $\hat{\chi}_e(\omega)$ and $\hat{\chi}_m(\omega)$ are the
electric and magnetic susceptibility, respectively.\footnote{Here we
  will assume that $\hat{\chi}_e(\omega)$ and $\hat{\chi}_m(\omega)$
  are not functions of time.  Thus frequency response of the material
  ``today'' is the same as it will be ``tomorrow.''}  Thus we can
write
\begin{eqnarray}
  \hDvec(\omega) &=& \epsilon_0\hEvec(\omega)
        	+ \epsilon_0\hat{\chi}_e(\omega)\hEvec(\omega), 
  \label{eq:dispersiveD} \\ 
  \hBvec(\omega) &=&\mu_0\hHvec(\omega)
	 + \mu_0\hat{\chi}_m(\omega)\hHvec(\omega). 
  \label{eq:dispersiveB}
\end{eqnarray}
For the time being we restrict discussion to the electric fields where
the permittivity $\hat{\epsilon}(\omega)$ is defined as
\begin{equation}
 \hat{\epsilon}(\omega) = \epsilon_0\hat{\epsilon}_r(\omega) =
     \epsilon_0(\epsilon_\infty + \hat{\chi}_e(\omega)).
\end{equation}
where $\hat{\epsilon}_r$ is the relative permittivity and, as will be
seen, the constant $\epsilon_\infty$ accounts for the effect of the
charged material at high frequencies where the susceptibility
function goes to zero.

The time-domain electric flux density can be obtain by inverse
transforming \refeq{eq:dispersiveD}.  The product of
$\hat{\chi}_e$ and $\hEvec$ in the frequency domain yields a
convolution in the time domain.  The fields are assumed to be zero
prior to $t=0$, so this yields
\begin{equation}
  \Dvec(t) = \epsilon_0\Evec(t) + \int_{\tau=0}^t
	  \chi_e(\tau)\Evec(t-\tau)d\tau
\end{equation}
where $\chi_e(t)$ is the inverse transform of $\hat{\chi}_e(\omega)$:
\begin{equation}
  \chi_e(t) =
  \frac{1}{2\pi}\int_{-\infty}^\infty \hat{\chi}_e(\omega) e^{j\omega t} d\omega.
\end{equation}
The time-domain function $\epsilon_0\chi_e(t)$ corresponds to the
polarization vector $\mathbf{P}(t)$ for an impulsive electric
field---effectively the impulse response of the medium.

We now consider three common susceptibility functions.  As will be
shown these are based on either simple mechanical or electrical
models.

%%%%%%%%%%%%%%%%%%%%%%%%%%%%%%%%%%%%%%%%%%%%%%%%%%%%%%%%%%%%%%%%%%%%%%
%%%%%%%%%%%%%%%%%%%%%%%%%%%%%%%%%%%%%%%%%%%%%%%%%%%%%%%%%%%%%%%%%%%%%%
%%%%%%%%%%%%%%%%%%%%%%%%%%%%%%%%%%%%%%%%%%%%%%%%%%%%%%%%%%%%%%%%%%%%%%
\subsection{Drude Materials}

In the Drude model, which often provides a good model of the behavior
of conductors, charges are assumed to move under the influence of the
electric field but they experience a damping force as well.  This can
be described by the following simple mechanical model
\begin{equation}
  M\frac{d^2\mathbf{x}}{dt^2} = Q\Evec(t) - Mg \frac{d\mathbf{x}}{dt}
  \label{eq:drudePhysical}
\end{equation}
where $M$ is the mass of the charge, $g$ is the damping coefficient,
$Q$ is the amount of charge, and $\mathbf{x}$ is the displacement of
the charge (the displacement is assumed to be in an arbitrary
direction and not restricted to the $x$ axis despite the use of the
symbol $\mathbf{x}$).  The left side of this equation is mass times
acceleration and the right side is the sum of the forces on the
charge, i.e., a driving force and a damping force.  Rearranging this
and converting to the frequency domain yields
\begin{equation}
  M(j\omega)^2\hxvec(\omega) + Mg
  (j\omega)\hxvec(\omega) = Q\hEvec(\omega).
\end{equation}
Thus the displacement can be expressed as
\begin{equation}
  \hxvec(\omega) = 
   -\frac{Q}{M\left(\omega^2 - jg\omega\right)}\hat{\Evec}(\omega).
   \label{eq:xVecDef}
\end{equation}
The polarization vector $\mathbf{P}$ is related to the dipole moment
of individual charges.  If $N$ is the number of dipoles per unit
volume, the polarization vector is given by
\begin{equation}
  \hPvec = NQ\hxvec.
  \label{eq:pVecDef}
\end{equation}
Note that $\mathbf{P}$ has units of charge per units area (C/m$^2$)
and thus, as would be expected from \refeq{eq:dAndEAndP}, has the same
units as electric flux.  Combining \refeq{eq:xVecDef} and
\refeq{eq:pVecDef} yields
\begin{equation}
  \hPvec(\omega) =
   - \epsilon_0\frac{\frac{NQ^2}{M\epsilon_0}}{\omega^2 - jg\omega}
     \hEvec(\omega).
\end{equation}
The electric susceptibility for Drude materials is thus given by
\begin{equation}
  \hat{\chi}_e(\omega) = - \frac{\omega^2_p}{\omega^2 - jg\omega}
  \label{eq:drudeChiOmega}
\end{equation}
where $\omega^2_p=NQ^2/(M\epsilon_0)$.  However, we are not overly
concerned with the specifics behind any one constant.  For example,
some authors may elect to combine the mass and damping coefficient
which were kept as separate quantities in \refeq{eq:drudePhysical}
(the product of the two dictates the damping force).  Regardless of
how the constants are defined, ultimately the Drude susceptibility
will take the form shown in \refeq{eq:drudeChiOmega}.

The relative permittivity for a Drude material can thus be written
\begin{equation}
  \hat{\epsilon}_r(\omega) = \epsilon_\infty 
    - \frac{\omega^2_p}{\omega^2 - jg\omega}
  \label{eq:drudeEpsR}.
\end{equation}
Note that as $\omega$ goes to infinity the relative permittivity
reduces to $\epsilon_\infty$.  Consider a rather special case in which
$\epsilon_\infty=1$ and $g=0$.  When $\omega=\omega_p/\sqrt{2}$ the
relative permittivity is $-1$.  It is possible, at least to some
extent, to construct a material which has not only this kind of
behavior for permittivity but also for the behavior for the
permeability, i.e., $\mu_r=\epsilon_r=-1$.  This kind of material,
which is known by various names including meta material,
double-negative material, backward-wave material, and left-handed
material, possesses many interesting properties.  Some properties of
these ``meta materials'' are also rather controversial as some people
have made claims that others dispute (such as the ability to construct
a ``perfect'' lens using a planar slab of this material).

The impulse response for the medium is the inverse Fourier transform of 
\refeq{eq:drudeChiOmega}:
\begin{equation}
  \chi_e(t) = \frac{\omega^2_p}{g}\left(1-e^{-gt}\right)u(t)
\end{equation}
where $u(t)$ is the unit step function.  The factor $g$ is seen to
determine the rate at which the response goes to zero, i.e., its
inverse is the relaxation time.  The factor $\omega_p$ is known as the
plasma frequency.

%%%%%%%%%%%%%%%%%%%%%%%%%%%%%%%%%%%%%%%%%%%%%%%%%%%%%%%%%%%%%%%%%%%%%%
%%%%%%%%%%%%%%%%%%%%%%%%%%%%%%%%%%%%%%%%%%%%%%%%%%%%%%%%%%%%%%%%%%%%%%
%%%%%%%%%%%%%%%%%%%%%%%%%%%%%%%%%%%%%%%%%%%%%%%%%%%%%%%%%%%%%%%%%%%%%%
\subsection{Lorentz Material}

Lorentz material is based on a second-order mechanical model of charge
motion.  In this case, in addition to a damping force, there is a
restoring force (effectively a spring force which wants to bring the
charge back to its initial position).  The sum of the forces can be
expressed as
\begin{equation}
  M\frac{d^2\mathbf{x}}{dt^2} = 
  Q\Evec(t) - Mg \frac{d\mathbf{x}}{dt} - MK \mathbf{x}.
  \label{eq:lorentzPhysical}
\end{equation}
The terms in commons with those in \refeq{eq:drudePhysical} as they
had in the case of the Drude model.  The additional term represents
the restoring force (which is proportional to the displacement) and is
scaled by the spring constant $K$.

Converting to the frequency domain and rearranging yields
\begin{equation}
  -M\omega^2\hxvec(\omega) + 
  jMg\omega\hxvec(\omega) + MK \hxvec(\omega)
  = Q\hEvec(\omega).
\end{equation}
Thus the displacement is given by 
\begin{equation}
  \hxvec(\omega) = 
   \frac{Q}{M\left(K + jg\omega - \omega^2 \right)}\hEvec(\omega).
\end{equation}
Relating displacement to the polarization vector via $\hPvec =
NQ\hxvec$ yields
\begin{equation}
  \hPvec(\omega) = 
   \epsilon_0\frac{NQ^2}
        {M\epsilon_0\left(K + jg\omega - \omega^2 \right)}
        \hEvec(\omega).
\end{equation}
From this the susceptibility is identified as
\begin{equation}
 \hat{\chi}_e(\omega) = 
     \frac{NQ^2}
        {M\epsilon_0\left(K + jg\omega - \omega^2 \right)}
\end{equation}
However, as before, the important thing is the form of the function,
not the individual constants.  We thus write this as
\begin{equation}
 \hat{\chi}_e(\omega) = 
     \frac{\epsilon_\ell\omega^2_\ell}
        {\omega^2_\ell + 2 j g_\ell \omega - \omega^2}
 \label{eq:lorentzOmega}
\end{equation}
where $\omega_\ell$ is the undamped resonant frequency, $g_\ell$ is
the damping coefficient, and $\epsilon_\ell$ (together with
$\epsilon_\infty$) accounts for the relatively permittivity at zero
frequency.  The corresponding relative permittivity is given by
\begin{equation}
   \hat{\epsilon}_r(\omega) = \epsilon_\infty + 
     \frac{\epsilon_\ell\omega^2_\ell}
        {\omega^2_\ell + 2 j g_\ell \omega - \omega^2}.
\end{equation}
Note that when the frequency goes to zero the relative permittivity
becomes $\epsilon_\infty + \epsilon_\ell$ whereas when the frequency
goes to infinity the relative permittivity is simply $\epsilon_\infty$.

The time-domain form of the susceptibility function is given by the
inverse transform of \refeq{eq:lorentzOmega}.  This yields
\begin{equation}
 \chi_e(t) = 
    \frac{\epsilon_\ell \omega^2_\ell}
    {\sqrt{\omega^2_\ell-g^2_\ell}} e^{-g_\ell t}
    \sin\!\left(t \sqrt{\omega^2_\ell-g^2_\ell}\right)u(t).
\end{equation}


%%%%%%%%%%%%%%%%%%%%%%%%%%%%%%%%%%%%%%%%%%%%%%%%%%%%%%%%%%%%%%%%%%%%%%
%%%%%%%%%%%%%%%%%%%%%%%%%%%%%%%%%%%%%%%%%%%%%%%%%%%%%%%%%%%%%%%%%%%%%%
%%%%%%%%%%%%%%%%%%%%%%%%%%%%%%%%%%%%%%%%%%%%%%%%%%%%%%%%%%%%%%%%%%%%%%
\subsection{Debye Material}

Debye materials can be thought of as a simple RC circuit where the
amount of polarization is related to the voltage across the capacitor.
The ``source'' driving the circuit is the electric field.  For a step
in the source, there may be a constant polarization.  As the frequency
goes to infinity, the polarization goes to zero (leaving just the
constant residual high-frequency term).  The susceptibility is thus
written
\begin{equation}
  \hat{\chi}_e(\omega) = \frac{\epsilon_d}{1+j\omega \tau_d}
\end{equation}
where $\tau_d$ is the time constant and $\epsilon_d$ (together
with $\epsilon_\infty$) accounts for the relative permittivity when
the frequency is zero.  The relative permittivity is given by
\begin{equation}
  \hat{\epsilon}_r(\omega) =
    \epsilon_\infty + \frac{\epsilon_d}{1+j\omega \tau_d}.
\end{equation}
The time-domain form of the susceptibility function is
\begin{equation}
  \chi_e(t) = \frac{\epsilon_d}{\tau_d} e^{-t/\tau_d}u(t).
  \label{eq:debyeChiTime}
\end{equation}

%%%%%%%%%%%%%%%%%%%%%%%%%%%%%%%%%%%%%%%%%%%%%%%%%%%%%%%%%%%%%%%%%%%%%%
%%%%%%%%%%%%%%%%%%%%%%%%%%%%%%%%%%%%%%%%%%%%%%%%%%%%%%%%%%%%%%%%%%%%%%
%%%%%%%%%%%%%%%%%%%%%%%%%%%%%%%%%%%%%%%%%%%%%%%%%%%%%%%%%%%%%%%%%%%%%%
\section{Debye Materials Using the ADE Method}

The finite-difference approximation of the differential equation that
relates the polarization and the electric field can be used to obtain
the polarization at future times in terms of its past value and an
expression involving the electric field.  By doing so, one can obtain
a consistent FDTD model that requires that a quantity related to the
polarization be stored as an additional variable.  This approach is
known as the auxiliary differential equation (ADE) method.

In the frequency domain Ampere's law can be written
\begin{equation}
 \epsilon_0\epsilon_\infty j\omega\hEvec+\sigma\hEvec+\hJpvec =
	\nabla\times\hHvec
\end{equation}
where the polarization current $\hJpvec$ is given by
\begin{equation}
  \hJpvec = j\omega\hPvec = j\omega\epsilon_0\hat{\chi}_e\hEvec.
\end{equation}
For a Debye material this becomes
\begin{equation}
  \hJpvec = j\omega\epsilon_0
	\frac{\epsilon_d}{1+j\omega\tau}\hEvec.
\end{equation}
Multiplying through by $1+j\omega\tau$ yields
\begin{equation}
  \hJpvec + j\omega\tau\hJpvec = j\omega\epsilon_0\epsilon_d\hEvec.
\end{equation}
Converting to the time domain produces
\begin{equation}
  \Jpvec + \tau\frac{\partial\Jpvec}{\partial t} 
   = \epsilon_0\epsilon_d\frac{\partial\Evec}{\partial t}.
\end{equation}
Discretizing this about the time-step $q+1/2$ yields
\begin{equation}
  \frac{\Jpvec^{q+1} + \Jpvec^q}{2} + 
  \tau\frac{\Jpvec^{q+1} - \Jpvec^q}{\Delt} =
  \epsilon_0\epsilon_d\frac{\Evec^{q+1}-\Evec^q}{\Delt}.
  \label{eq:vectorDiscreteJp}
\end{equation}
Since we are assuming an isotropic medium, the polarization current is
aligned with the electric field.  Hence in \refeq{eq:vectorDiscreteJp}
the $x$ component of $\Jpvec$ depends only on the $x$ component of
$\Evec$ (and similarly for the $y$ and $z$ components).

Solving \refeq{eq:vectorDiscreteJp} for $\Jpvec^{q+1}$ yields
\begin{equation}
  \Jpvec^{q+1} =
    \frac{1-\frac{\Delt}{2\tau}}{1+\frac{\Delt}{2\tau}}\Jpvec^q +
    \frac{\frac{\Delt}{\tau}}{1+\frac{\Delt}{2\tau}}
    \frac{\epsilon_0\epsilon_d}{\Delt}
    \left(\Evec^{q+1}-\Evec^q\right).
    \label{eq:jpUpdateDebyeI}
\end{equation}
Whatever the  time-constant $\tau$ is, it can be expressed in terms of
some multiple of the time-step, i.e., $\tau=N_\tau\Delt$ where
$N_\tau$ does not need to be an integer.  As will be shown, it is
convenient to multiply both sides of \refeq{eq:jpUpdateDebyeI} by the
spatial step size.  We will assume a uniform grid in which
$\Delx=\Dely=\Delz=\delta$.  Thus \refeq{eq:jpUpdateDebyeI} can be
written
\begin{equation}
  \delta\Jpvec^{q+1} =
    \frac{1-\frac{1}{2N_\tau}}{1+\frac{1}{2N_\tau}}\delta\Jpvec^q +
    \frac{\frac{1}{N_\tau}}{1+\frac{1}{2N_\tau}}
    \frac{\epsilon_0\epsilon_d\delta}{\Delt}\left(\Evec^{q+1}-\Evec^q\right).
    \label{eq:jpUpdateDebyeII}
\end{equation}
Consider the factor $\epsilon_0\epsilon_d\delta/\Delt$:
\begin{equation}
  \frac{\epsilon_0\epsilon_d\delta}{\Delt} =
  \frac{\sqrt{\epsilon_0\mu_0}\epsilon_d\delta}{\sqrt{\frac{\mu_0}{\epsilon_0}}\Delt}
  =
  \frac{\epsilon_d\delta}{\eta_0 c\Delt} =
  \frac{\epsilon_d}{\eta_0 S_c}.
\end{equation}
Therefore \refeq{eq:jpUpdateDebyeII} can be written
\begin{equation}
  \delta\Jpvec^{q+1} =
    \cjj\delta\Jpvec^q +
    \cje\left(\Evec^{q+1}-\Evec^q\right),
    \label{eq:jpUpdateDebyeIII}
\end{equation}
where
\begin{eqnarray}
  \cjj &=& \frac{1-\frac{1}{2N_\tau}}{1+\frac{1}{2N_\tau}}, \\
  \cje &=& \frac{\frac{1}{N_\tau}}{1+\frac{1}{2N_\tau}}
             \frac{\epsilon_d}{\eta_0 S_c}.
\end{eqnarray}
Recall that Ampere's law is discretized about the time-step
$(q+1/2)\Delt$.  Since the polarization current appears in Ampere's
law, we thus need an expression for $\delta\Jpvec^{q+1/2}$.  This is
simply given by the average of $\delta\Jpvec^{q+1}$ (which is given by
\refeq{eq:jpUpdateDebyeIII}) and $\delta\Jpvec^{q}$:
\begin{equation}
  \delta\Jpvec^{q+1/2} =
  \frac{\delta\Jpvec^{q+1}+\delta\Jpvec^{q}}{2} = 
    \frac{1}{2}\left([1+\cjj]\delta\Jpvec^q +
    \cje\left(\Evec^{q+1}-\Evec^q\right)\right).
  \label{eq:jpDebyeHalf}
\end{equation}

The discrete time-domain form of Ampere's law expanded about the
time-step $(q+1/2)$ is
\begin{equation}
 \epsilon_0\epsilon_\infty \frac{\Evec^{q+1}-\Evec^q}{\Delt}
 +\sigma\frac{\Evec^{q+1}+\Evec^q}{2}+\Jpvec^{q+1/2} =
	\nabla\times\Hvec^{q+1/2}.
 \label{eq:ampereDiscrete}
\end{equation}
Multiplying through by $\delta$ and using \refeq{eq:jpDebyeHalf} in 
\refeq{eq:ampereDiscrete} yields
\begin{equation}
 \frac{\epsilon_\infty\epsilon_0\delta}{\Delt} 
 \left(\Evec^{q+1}-\Evec^q\right)
 +\frac{\sigma\delta}{2}\left(\Evec^{q+1}+\Evec^q\right)
 + \frac{1}{2}\left([1+\cjj]\delta\Jpvec^q +
    \cje\left(\Evec^{q+1}-\Evec^q\right)\right)=
	\delta\nabla\times\Hvec^{q+1/2}.
 \label{eq:ampereDiscreteDebyeI}
\end{equation}
The curl of the magnetic field, which involves spatial derivatives,
will have a $\delta$ in the denominator of the finite differences.
Thus $\delta\nabla\times\Hvec^{q+1/2}$ will involve merely the
difference of the various magnetic-field components.

Regrouping terms in \refeq{eq:ampereDiscreteDebyeI} produces
\begin{equation}
 \Evec^{q+1}\left(\frac{\epsilon_\infty\epsilon_0\delta}{\Delt} 
                   + \frac{\sigma\delta}{2}
                   + \frac{1}{2}\cje\right)
 =
 \Evec^q\left(\frac{\epsilon_\infty\epsilon_0\delta}{\Delt} 
                   - \frac{\sigma\delta}{2}
                   + \frac{1}{2}\cje\right)
 +\delta\nabla\times\Hvec^{q+1/2} -
 \frac{1}{2}\left[1+\cjj\right]\delta\Jpvec^q.
 \label{eq:eUpdateDebye}
\end{equation}
The term multiplying the electric fields can be written as
\begin{equation}
  \frac{\epsilon_\infty\epsilon_0\delta}{\Delt}\left(1
                   \pm \frac{\sigma\Delt}{2\epsilon_\infty\epsilon_0}
                   + \frac{\cje\Delt}{2\epsilon_\infty\epsilon_0\delta}\right)
  =
  \frac{\epsilon_\infty}{\eta_0 S_c}\left(1
                   \pm \frac{\sigma\Delt}{2\epsilon_\infty\epsilon_0}
                   + \frac{\cje\eta_0 S_c}{2\epsilon_\infty}\right).
\end{equation}
Therefore \refeq{eq:eUpdateDebye} can be written
\begin{equation}
 \Evec^{q+1}=
  \frac{1 - \frac{\sigma\Delt}{2\epsilon_\infty\epsilon_0}
          + \frac{\cje\eta_0 S_c}{2\epsilon_\infty}}
       {1 + \frac{\sigma\Delt}{2\epsilon_\infty\epsilon_0}
          + \frac{\cje\eta_0 S_c}{2\epsilon_\infty}}
 \Evec^q
 +  \frac{\frac{\eta_0 S_c}{\epsilon_\infty}}{1
                   + \frac{\sigma\Delt}{2\epsilon_\infty\epsilon_0}
                   + \frac{\cje\eta_0 S_c}{2\epsilon_\infty}}
\left(\delta\nabla\times\Hvec^{q+1/2} -
 \frac{1}{2}\left[1+\cjj\right]\delta\Jpvec^q\right).
 \label{eq:eUpdateDebyeI}
\end{equation}
The way in which the term $\sigma\Delt/(2\epsilon_0)$ could be
expressed in terms of the skin depth was discussed in Sec.\
\ref{sec:conductivity}.

The FDTD model of a Debye material would be implemented as follows:
\begin{enumerate}
\item Update the magnetic fields in the usual way.  
      $\Hvec^{q-1/2}\Rightarrow \Hvec^{q+1/2}$.
\item For each electric-field component, do the following:
\begin{enumerate}
\item Copy the electric field to a temporary variable.
      $E_{\mathit{tmp}}=E^q$.
\item Update the electric field using \refeq{eq:eUpdateDebyeI}.
      $E^{q}\Rightarrow E^{q+1}$.
\item Update the polarization current (actually $\delta$ times the
      polarization current) using
      \refeq{eq:jpUpdateDebyeIII}. $J_p^q\Rightarrow J_p^{q+1}$.
      This update requires both $E^{q+1}$ and $E^q$ (which is stored
      in $E_{\mathit{tmp}}$).
\end{enumerate}
\item Repeat.
\end{enumerate}
The polarization current (actually the product of the spatial
step-size and the polarization current, i.e., $\delta\Jpvec$) must be
stored as a separate quantity.  Of course it only needs to be stored
for nodes at which it is non-zero.

If the polarization current is initially zero and $\cje$ is zero, the
polarization current will always be zero and the material behaves as a
standard non-dispersive material (although, of course, dispersion is
always present in the grid itself).  Thus the update equations
presented here can be used throughout a simulation which is a mix of
dispersive and non-dispersive media.  One merely has to set the
constants to the appropriate values for a given location.

%%%%%%%%%%%%%%%%%%%%%%%%%%%%%%%%%%%%%%%%%%%%%%%%%%%%%%%%%%%%%%%%%%%%%%
%%%%%%%%%%%%%%%%%%%%%%%%%%%%%%%%%%%%%%%%%%%%%%%%%%%%%%%%%%%%%%%%%%%%%%
%%%%%%%%%%%%%%%%%%%%%%%%%%%%%%%%%%%%%%%%%%%%%%%%%%%%%%%%%%%%%%%%%%%%%%
\section{Drude Materials Using the ADE Method}

The electric susceptibility for a Drude material is given in
\refeq{eq:drudeChiOmega} and can also be written
\begin{equation}
  \hat{\chi}_e(\omega) =  \frac{\omega^2_p}{j\omega(j\omega + g)}.
\end{equation}
The associated polarization current is
\begin{equation}
  \hJpvec = j\omega\hPvec = j\omega\epsilon_0\hat{\chi}_e\hEvec = 
  j\omega\epsilon_0\frac{\omega_p^2}{j\omega(j\omega + g)}\hEvec.
\end{equation}
Canceling $j\omega$ and cross multiplying by $g+j\omega$ yields
\begin{equation}
g\hJpvec + j\omega\hJpvec = \epsilon_0\omega_p^2\hEvec.
\end{equation}
Expressed in the time-domain, this is
\begin{equation}
  g\Jpvec + \frac{\partial\Jpvec}{\partial t} =
  \epsilon_0\omega_p^2\Evec.
\end{equation}
Discretizing time and expanding this about the time-step $q+1/2$
yields
\begin{equation}
  g\frac{\Jpvec^{q+1} +\Jpvec^q}{2}+ \frac{\Jpvec^{q+1}-\Jpvec^q}{\Delt} =
  \epsilon_0\omega_p^2\frac{\Evec^{q+1}+\Evec^q}{2}.
\end{equation}
Solving for $\Jpvec^{q+1}$ we obtain
\begin{equation}
  \Jpvec^{q+1} = \frac{1-\frac{g\Delt}{2}}{1+\frac{g\Delt}{2}}\Jpvec^q
  +\frac{1}{1+\frac{g\Delt}{2}}
   \frac{\epsilon_0\omega_p^2\Delt}{2}\left(\Evec^{q+1}+\Evec^q\right).
  \label{eq:jpUpdateDrude}
\end{equation}
The damping or loss term $g$ is the inverse of the relaxation time and
hence can be expressed as a multiple of the number of time steps,
i.e.,
\begin{equation}
  g = \frac{1}{N_g\Delt}
\end{equation}
where $N_g$ does not need to be integer.  Consider the term
multiplying the electric field
\begin{equation}
  \frac{\epsilon_0\omega_p^2\Delt}{2} = 
  \frac{\sqrt{\epsilon_0\mu_0}4 \pi^2 f_p^2 \Delt}
       {\sqrt{\frac{\mu_0}{\epsilon_0}}2} = 
  \frac{2\pi^2 c^2 \Delt}{\eta_0\lambda_p^2c} = 
  \frac{2\pi^2 c \Delt}{\eta_0(N_p\delta)^2} = 
  \frac{2\pi^2 S_c}{\eta_0 N_p^2 \delta}
\end{equation}
where $f_p$ is the plasma frequency in Hertz, $\lambda_p$ is the
free-space wavelength at this frequency, and $N_p$ is the number of
points per wavelength at the plasma frequency.  Since this term
contains $\delta$ in the denominator, we multiply
\refeq{eq:jpUpdateDrude} by $\delta$ to obtain
\begin{equation}
  \delta\Jpvec^{q+1} = \cjj\delta\Jpvec^q
  +\cje\left(\Evec^{q+1}+\Evec^q\right)
  \label{eq:jpUpdateDrudeI}
\end{equation}
where
\begin{eqnarray}
  \cjj &=& \frac{1-\frac{1}{2N_g}}{1+\frac{1}{2N_g}},  \\
  \cje &=& \frac{1}{1+\frac{1}{2N_g}}\frac{2\pi^2 S_c}{\eta_0 N_p^2}.
\end{eqnarray}
Note the similarity between \refeq{eq:jpUpdateDebyeIII} and
\refeq{eq:jpUpdateDrudeI}.  These equations have nearly identical
forms but the constants are different and there is a different sign
associated with the ``old'' value of the electric field.

The general discretized form of Ampere's law expanded about the
time-step $(q+1/2)$ is unchanged from \refeq{eq:ampereDiscrete} and is
repeat below
\begin{equation}
 \epsilon_0\epsilon_\infty \frac{\Evec^{q+1}-\Evec^q}{\Delt}
 +\sigma\frac{\Evec^{q+1}+\Evec^q}{2}+\Jpvec^{q+1/2} =
	\nabla\times\Hvec^{q+1/2}.
\end{equation}
As before, $\delta\Jpvec^{q+1/2}$ can be obtained by the average of
$\delta\Jpvec^{q+1}$ and $\delta\Jpvec^{q}$:
\begin{equation}
  \delta\Jpvec^{q+1/2} =
  \frac{\delta\Jpvec^{q+1}+\delta\Jpvec^{q}}{2} = 
    \frac{1}{2}\left([1+\cjj]\delta\Jpvec^q +
    \cje\left(\Evec^{q+1}+\Evec^q\right)\right).
  \label{eq:jpDrudeHalf}
\end{equation}
Multiplying through by $\delta$ and using \refeq{eq:jpDrudeHalf} for
the polarization current yields
\begin{equation}
 \frac{\epsilon_\infty\epsilon_0\delta}{\Delt} 
 \left(\Evec^{q+1}-\Evec^q\right)
 + \frac{\sigma\delta}{2}\left(\Evec^{q+1}+\Evec^q\right)
 + \frac{1}{2}\left([1+\cjj]\delta\Jpvec^q +
     \cje\left(\Evec^{q+1}+\Evec^q\right)\right)=
	\delta\nabla\times\Hvec^{q+1/2}.
\end{equation}
Note that, other than the constants being those for a Drude material,
the only way in which this expression differs from
\refeq{eq:ampereDiscreteDebyeI} is the sign of the ``old'' electric
field associated with the polarization current.  Therefore
\refeq{eq:eUpdateDebyeI} again yields the expression for the
``future'' electric field if we make the appropriate change of sign.
The result is
\begin{equation}
 \Evec^{q+1}=
  \frac{1 - \frac{\sigma\Delt}{2\epsilon_\infty\epsilon_0}
          - \frac{\cje\eta_0 S_c}{2\epsilon_\infty}}
       {1 + \frac{\sigma\Delt}{2\epsilon_\infty\epsilon_0}
          + \frac{\cje\eta_0 S_c}{2\epsilon_\infty}}
 \Evec^q
 +  \frac{\frac{\eta_0 S_c}{\epsilon_\infty}}{1
                   + \frac{\sigma\Delt}{2\epsilon_\infty\epsilon_0}
                   + \frac{\cje\eta_0 S_c}{2\epsilon_\infty}}
\left(\delta\nabla\times\Hvec^{q+1/2} -
 \frac{1}{2}\left[1+\cjj\right]\delta\Jpvec^q\right).
\label{eq:eUpdateDrude}
\end{equation}

The implementation of an FDTD algorithm for Drude material now
parallels that for Debye material:
\begin{enumerate}
\item Update the magnetic fields in the usual way.  
      $\Hvec^{q-1/2}\Rightarrow \Hvec^{q+1/2}$.
\item For each electric-field component, do the following
\begin{enumerate}
\item Copy the electric field to a temporary variable.
      $E_{\mathit{tmp}}=E^q$.
\item Update the electric field using \refeq{eq:eUpdateDrude}.
      $E^{q}\Rightarrow E^{q+1}$.
\item Update the polarization current (actually $\delta$ times the
      polarization current) using
      \refeq{eq:jpUpdateDrudeI}. $J^{q}\Rightarrow J^{q+1}$.
      This update requires both $E^{q+1}$ and $E^q$ (which is stored
      in $E_{\mathit{tmp}}$).
\end{enumerate}
\item Repeat.
\end{enumerate}

%%%%%%%%%%%%%%%%%%%%%%%%%%%%%%%%%%%%%%%%%%%%%%%%%%%%%%%%%%%%%%%%%%%%%%
%%%%%%%%%%%%%%%%%%%%%%%%%%%%%%%%%%%%%%%%%%%%%%%%%%%%%%%%%%%%%%%%%%%%%%
%%%%%%%%%%%%%%%%%%%%%%%%%%%%%%%%%%%%%%%%%%%%%%%%%%%%%%%%%%%%%%%%%%%%%%
\section{Magnetically Dispersive Material}

In the frequency domain Faraday's law is
\begin{equation}
  \nabla\times\hEvec = -j\omega\hBvec - \sigma_m\hHvec
\end{equation}
where $\sigma_m$ is the magnetic conductivity.  Generalizing
\refeq{eq:dispersiveB} slightly, the permeability can be written
as
\begin{equation}
 \hat{\mu}(\omega) = \mu_0\hat{\mu}_r(\omega) =
     \mu_0(\mu_\infty + \hat{\chi}_m(\omega)).
\end{equation}
where the factor $\mu_\infty$ accounts for the permeability at high
frequencies.  Faraday's law can thus be written
\begin{equation}
  \nabla\times\hEvec = -j\omega\mu_0\mu_\infty\hHvec(\omega)
	 - \sigma_m\hHvec
         - \hJmvec.
\end{equation}
where the magnetic polarization current $\hJmvec$ is given by
\begin{equation}
  \hJmvec = j\omega\mu_0\hMvec = 
  j\omega\mu_0\hat{\chi}_m(\omega)\hHvec(\omega).
\end{equation}

The derivation of the equations which govern an FDTD implementation of
a magnetically dispersive material parallel that of electrically
dispersive material.  Here we consider the case of Drude dispersion
where
\begin{equation}
  \hat{\chi}_m(\omega) = -\frac{\omega_p^2}{\omega^2 -jg\omega}
  =  \frac{\omega_p^2}{j\omega(j\omega + g)}.
\end{equation}
(The plasma frequency $\omega_p$ and damping term $g$ for magnetic
susceptibility are distinct from those of electric susceptibility.)
Thus the polarization current and magnetic field are related by
\begin{equation}
  \hJmvec = j\omega\mu_0\frac{\omega_p^2}{j\omega(j\omega + g)}\hHvec
  = \frac{\mu_0\omega_p^2}{(j\omega + g)}\hHvec.
\end{equation}
Multiplying by $g+j\omega$ yields
\begin{equation}
  (g+j\omega)\hJmvec = \mu_0\omega_p^2\hHvec.
\end{equation}
The time-domain equivalent of this is
\begin{equation}
  g\Jmvec+\frac{\partial\Jmvec}{\partial t} = \mu_0\omega_p^2\Hvec.
\end{equation}
Discretizing time and expanding this about the time-step $q$ yields
\begin{equation}
  g\frac{\Jmvec^{q+1/2}+\Jmvec^{q-1/2}}{2}
  +\frac{\Jmvec^{q+1/2}-\Jmvec^{q-1/2}}{\Delt} = 
  \mu_0\omega_p^2\frac{\Hvec^{q+1/2}+\Hvec^{q-1/2}}{2}.
\end{equation}
Solving for $\Jmvec^{q+1/2}$ yields
\begin{equation}
  \Jmvec^{q+1/2} = \frac{1-\frac{g\Delt}{2}}{1+\frac{g\Delt}{2}}\Jmvec^{q-1/2}
  + \frac{1}{1+\frac{g\Delt}{2}}
   \frac{\mu_0\omega_p^2\Delt}{2}\left(\Hvec^{q+1/2}+\Hvec^{q-1/2}\right).
  \label{eq:jmUpdateDrude}
\end{equation}
Consider the term multiplying the magnetic field
\begin{equation}
  \frac{\mu_0\omega_p^2\Delt}{2} = 
  \frac{\sqrt{\frac{\mu_0}{\epsilon_0}}
        \sqrt{\epsilon_0\mu_0}4 \pi^2 f_p^2 \Delt}{2} = 
  \frac{2\pi^2 \eta_0 c^2 \Delt}{\lambda_p^2c} =
  \frac{2\pi^2 \eta_0 c \Delt}{(N_p\delta)^2} = 
  \frac{2\pi^2 \eta_0 S_c}{N_p^2 \delta}.
  \label{eq:coefficientDrudeM}
\end{equation}
where, as before, $f_p$ is the plasma frequency in Hertz, $\lambda_p$
is the free-space wavelength at this frequency, and $N_p$ is the
number of points per wavelength at the plasma frequency.  Again
expressing the damping coefficient in terms of some multiple of the
time step, i.e., $g=1/(N_g\Delt)$, multiplying all terms in
\refeq{eq:jmUpdateDrude} by $\delta$, and employing the final form of the
term given in \refeq{eq:coefficientDrudeM}, the update equation for
the polarization current can be written as
\begin{equation}
  \delta\Jmvec^{q+1/2} = \cjj\delta\Jmvec^{q-1/2}
  +\cjh\left(\Hvec^{q+1/2}+\Hvec^{q-1/2}\right)
  \label{eq:jmUpdateDrudeI}
\end{equation}
where
\begin{eqnarray}
  \cjj &=& \frac{1-\frac{1}{2N_g}}{1+\frac{1}{2N_g}},  \\
  \cjh &=& \frac{1}{1+\frac{1}{2N_g}}\frac{2\pi^2 \eta_0 S_c}{N_p^2}.
\end{eqnarray}
To obtain the magnetic polarization current at time-step $q$, the
current at time-steps $q+1/2$ and $q-1/2$ are averaged:
\begin{equation}
  \delta\Jmvec^q =
  \frac{\delta\Jmvec^{q+1/2}+\delta\Jmvec^{q-1/2}}{2} = 
    \frac{1}{2}\left([1+\cjj]\delta\Jmvec^{q-1/2} +
    \cjh\left(\Hvec^{q+1/2}+\Hvec^{q-1/2}\right)\right).
  \label{eq:jmDrudeQ}
\end{equation}

The discretized form of Faraday's law expanded about time-step $q$ is
\begin{equation}
 -\mu_0\mu_\infty\frac{\Hvec^{q+1/2}-\Hvec^{q-1/2}}{\Delt}
 - \sigma_m\frac{\Hvec^{q+1/2}-\Hvec^{q-1/2}}{2} - \Jmvec^q =
	\nabla\times\Evec^q.
\end{equation}
Multiplying through by $-\delta$ and using \refeq{eq:jmDrudeQ} for
the polarization current yields
\begin{eqnarray}
 \lefteqn{\frac{\mu_\infty\mu_0\delta}{\Delt} 
 \left(\Hvec^{q+1/2}-\Hvec^{q-1/2}\right)
 +\frac{\sigma_m\delta}{2}\left(\Hvec^{q+1/2}+\Hvec^{q-1/2}\right)} 
 \nonumber\\
 && \mbox{} + \frac{1}{2}\left([1+\cjj]\delta\Jmvec^{q-1/2} +
    \cjh\left(\Hvec^{q+1/2}+\Hvec^{q-1/2}\right)\right)=
	-\delta\nabla\times\Evec^q.
 \label{eq:faradayDrude}
\end{eqnarray}
After regrouping terms we obtain
\begin{eqnarray}
 \lefteqn{\Hvec^{q+1/2}\left(\frac{\mu_\infty\mu_0\delta}{\Delt} 
                   + \frac{\sigma_m\delta}{2}
                   + \frac{1}{2}\cjh\right) =} \nonumber\\
 &&\Hvec^{q-1/2}\left(\frac{\mu_\infty\mu_0\delta}{\Delt} 
                   - \frac{\sigma_m\delta}{2}
                   + \frac{1}{2}\cjh\right)
 -\delta\nabla\times\Evec^q -
 \frac{1}{2}\left[1+\cjj\right]\delta\Jmvec^{q-1/2}.
 \label{eq:hUpdateDrude}
\end{eqnarray}
The factor $\mu_0\delta/\Delt$ is equivalent to $\eta_0/S_c$ so that
the update equation for the magnetic field can be written
\begin{equation}
 \Hvec^{q+1/2}=
  \frac{1 - \frac{\sigma_m\Delt}{2\mu_\infty\mu_0}
          - \frac{\cjh S_c}{2\eta_0 \mu_\infty}}
       {1 + \frac{\sigma_m\Delt}{2\mu_\infty\mu_0}
          + \frac{\cjh S_c}{2\eta_0 \mu_\infty}} \Hvec^{q-1/2}
 + \frac{\frac{S_c}{\eta_0 \mu_\infty}}{1
                   + \frac{\sigma_m\Delt}{2\mu_\infty\mu_0}
                   + \frac{\cjh S_c}{2\eta_0 \mu_\infty}}
 \left(-\delta\nabla\times\Evec^q -
       \frac{1}{2}\left[1+\cjj\right]\delta\Jmvec^{q-1/2}\right).
 \label{eq:hUpdateDrudeI}
\end{equation}

An FDTD model of a magnetically dispersive material would be
implemented as follows:
\begin{enumerate}
\item For each magnetic-field component, do the following
\begin{enumerate}
\item Copy the magnetic field to a temporary variable.
      $H_{\mathit{tmp}}=H^{q-1/2}$.
\item Update the magnetic field using \refeq{eq:hUpdateDrudeI}.
      $H^{q-1/2}\Rightarrow H^{q+1/2}$.
\item Update the polarization current (actually $\delta$ times the
      polarization current) using
      \refeq{eq:jmUpdateDrudeI}. $J_m^{q-1/2}\Rightarrow J_m^{q+1/2}$.
      This update requires both $H^{q+1/2}$ and $H^{q-1/2}$ (which is stored
      in $H_{\mathit{tmp}}$).
\end{enumerate}
\item Update the electric fields in whatever way is appropriate (which
      may include the dispersive implementations described previously)
      $\Evec^{q}\Rightarrow \Evec^{q+1}$.
\item Repeat.
\end{enumerate}

\section{Piecewise Linear Recursive Convolution}

An alternative implementation of dispersive material is offered by the 
piecewise linear recursive convolution (PLRC) method.  Recall that
multiplication in the frequency domain is equivalent to convolution in
the time domain.  Thus, the frequency-domain relationship
\begin{equation}
  \hDvec(\omega) = \epsilon_0\epsilon_\infty\hEvec(\omega) + 
    \epsilon_0\hat{\chi}_e(\omega)\hEvec(\omega)
\end{equation}
is equivalent to the time-domain relationship
\begin{equation}
  \Dvec(t) = \epsilon_0\epsilon_\infty\Evec(t) + 
    \epsilon_0\int_{\zeta=0}^t\Evec(t-\zeta)\chi_e(\zeta)d\zeta
\end{equation}
where $\zeta$ is a dummy variable of integration.  In discrete form
the electric flux density at time-step $q\Delt$ can be written
\begin{equation}
  \Dvec^q = \epsilon_0\epsilon_\infty\Evec^q + 
    \epsilon_0\int_{\zeta=0}^{q\Delt}\Evec(q\Delt-\zeta)\chi_e(\zeta)d\zeta.
  \label{eq:plrcIntegral}
\end{equation}

Although in an FDTD simulation the electric field $\Evec$ would only
be available at discrete points in time, we wish to treat the field as
if it varies continuously insofar as the integral is concerned.
This is accomplished by assuming the field varies linearly between
sample points.  For example, assume the continuous variable $t$ is
between $i\Delt$ and $(i+1)\Delt$.  Over this range the electric field
is approximated by
\begin{equation}
  \Evec(t) = \Evec^i + \frac{\Evec^{i+1} - \Evec^{i}}{\Delt}(t-i\Delt)
  \quad \mbox{for }\quad  i\Delt\leq t\leq (i+1)\Delt.
\end{equation}
When $t$ is equal to $i\Delt$ we obtain $\Evec^i$ and when 
$t$ is $(i+1)\Delt$ we obtain $\Evec^{i+1}$.  The field varies
linearly between these points.

To obtain a more general representation of the electric field, let us
define the pulse function $p_i(t)$ which is given by
\begin{equation}
  p_i(t) = \left\{
  \begin{array}{l}
    1\quad \mbox{if} \quad i\Delt\leq t < (i+1)\Delt,\\
    0\quad \mbox{otherwise}.
  \end{array}\right.
\end{equation}
Using this pulse function the electric field can be written as
\begin{equation}
  \Evec(t) = \sum_{i=0}^{M-1}
    \left[\Evec^i +
          \frac{\Evec^{i+1} - \Evec^{i}}{\Delt}(t-i\Delt)\right] p_i(t).
\end{equation}
This provides a piecewise-linear approximation of the electric field
over $M$ segments.  Despite the summation, the pulse function ensures
that only one segment is turned on for any given value of $t$.  Hence
the summation can be thought of as serving more to collect together
the various segments rather than as serving to add several terms.

In the integrand of \refeq{eq:plrcIntegral} the argument of the
electric field is $q\Delt - \zeta$ where $q\Delt$ is constant with
respect to the variable of integration $\zeta$.  When $\zeta$ varies
from $i\Delt$ to $(i+1)\Delt$ the electric field should vary from the
discrete points $\Evec^{q-i}$ to $\Evec^{q-i-1}$.  Thus, the electric
field can be represented by
\begin{equation}
  \Evec(q\Delt-\zeta) = \Evec^{q-i} +
  \frac{\Evec^{q-i-1} - \Evec^{q-i}}{\Delt}(\zeta-i\Delt)
  \quad \mbox{for }\quad i\Delt\leq \zeta\leq (i+1)\Delt.
\end{equation}
Using the pulse function, the field over all values of $\zeta$ can be
written
\begin{equation}
  \Evec(q\Delt-\zeta) = \sum_{i=0}^{q-1}
    \left[\Evec^{q-i} +
          \frac{\Evec^{q-i-1}-\Evec^{q-i}}{\Delt}(\zeta-i\Delt)\right]
         p_i(\zeta).
  \label{eq:plrcESegments}
\end{equation}
Note the limits of the summation.  The upper limit of integration is
$q\Delt$ which corresponds to the end-point of the segment which
varies from $(q-1)\Delt$ to $q\Delt$.  This segment has an index of
$q-1$ (segment $0$ varies from $0$ to $\Delt$, segment $1$ varies from
$\Delt$ to $2\Delt$, and so on).  The lower limit used here is not
actually dictated by the electric field.  Rather, when we combine the
electric field with the susceptibility function the product is zero
for $\zeta$ less than zero since the susceptibility is zero for
$\zeta$ less than zero (due to the material impulse response being
causal).  Hence we start the lower limit of the summation at zero.

Substituting \refeq{eq:plrcESegments} into \refeq{eq:plrcIntegral}
yields
\begin{equation}
  \Dvec^q = \epsilon_0\epsilon_\infty\Evec^q + 
    \epsilon_0\int_{\zeta=0}^{q\Delt}\sum_{i=0}^{q-1}
    \left[\Evec^{q-i} +
          \frac{\Evec^{q-i-1}-\Evec^{q-i}}{\Delt}(\zeta-i\Delt)\right]
     p_i(\zeta)
    \chi_e(\zeta)d\zeta.
\end{equation}
The summation and integration can be interchanged.  However, the
pulse function dictates that the integration only needs to be carried
out over the range of values where the pulse function is unity.  Thus
we can write
\begin{equation}
  \Dvec^q = \epsilon_0\epsilon_\infty\Evec^q + 
    \epsilon_0\sum_{i=0}^{q-1}\int_{\zeta=i\Delt}^{(i+1)\Delt}\left[
    \Evec^{q-i} +
          \frac{\Evec^{q-i-1}-\Evec^{q-i}}{\Delt}(\zeta-i\Delt)\right]
    \chi_e(\zeta)d\zeta.
\end{equation}
The samples of the electric field are constants with respect to the
variable of integration and can be taken outside of the integral.
This yields
\begin{eqnarray}
  \lefteqn{\Dvec^q = \epsilon_0\epsilon_\infty\Evec^q +\mbox{}}  \nonumber \\
  &&
    \mbox{}\hspace{-.1in}\epsilon_0\sum_{i=0}^{q-1}\left[
    \Evec^{q-i} 
    \left(\int_{\zeta=i\Delt}^{(i+1)\Delt}\chi_e(\zeta)d\zeta\right)
    + \frac{\Evec^{q-i-1}-\Evec^{q-i}}{\Delt}
      \left(\int_{\zeta=i\Delt}^{(i+1)\Delt}(\zeta-i\Delt)\chi_e(\zeta)d\zeta\right)\right].
\end{eqnarray}
To simplify the notation, we define the following
\begin{eqnarray}
  \chi^i &=& \int_{\zeta=i\Delt}^{(i+1)\Delt}\chi_e(\zeta)d\zeta,
  \label{eq:ChiIDef}
 \\
  \xi^i &=&
     \frac{1}{\Delt}
     \int_{\zeta=i\Delt}^{(i+1)\Delt}(\zeta-i\Delt)\chi_e(\zeta)d\zeta.
  \label{eq:XiIDef}
\end{eqnarray}
This allows us to write
\begin{equation}
  \Dvec^q = \epsilon_0\epsilon_\infty\Evec^q + 
    \epsilon_0\sum_{i=0}^{q-1}\left[
    \Evec^{q-i} \chi^i
    + \left(\Evec^{q-i-1}-\Evec^{q-i}\right)\xi^i\right].
  \label{eq:plrcDq}
\end{equation}

In discrete form and using the electric flux density, Ampere's law can
be written
\begin{equation}
  \nabla\times\Hvec^{q+1/2} = \frac{\Dvec^{q+1} - \Dvec^{q}}{\Delt}.
  \label{eq:plrcAmpere}
\end{equation}
Equation \refeq{eq:plrcDq} gives $\Dvec^{q}$ in terms of the electric
field.  This equation can also be used to express $\Dvec^{q+1}$ in
terms of the electric field: one merely replaces $q$ with $q+1$.  This
yields
\begin{equation}
  \Dvec^{q+1} = \epsilon_0\epsilon_\infty\Evec^{q+1} + 
    \epsilon_0\sum_{i=0}^{q}\left[
    \Evec^{q-i+1} \chi^i
    + \left(\Evec^{q-i}-\Evec^{q-i+1}\right)\xi^i\right].
  \label{eq:plrcDqI}
\end{equation}
In order to obtain an update equation for the electric field, we must
express $\Evec^{q+1}$ in terms of other known (or past) quantities.
As things stand now, there is an $\Evec^{q+1}$ ``buried'' inside the
the summation in \refeq{eq:plrcDqI}.  To express that explicitly, we
extract the $i=0$ term and then have the summation start from $i=1$.
This yields
\begin{eqnarray}
  \Dvec^{q+1} &=& \epsilon_0\epsilon_\infty\Evec^{q+1} + 
    \epsilon_0 \Evec^{q+1} \chi^0 + \epsilon_0
    \left(\Evec^{q}-\Evec^{q+1}\right)\xi^0 \nonumber\\
 && +
    \epsilon_0\sum_{i=1}^{q}\left[
    \Evec^{q-i+1} \chi^i
    + \left(\Evec^{q-i}-\Evec^{q-i+1}\right)\xi^i\right].
  \label{eq:plrcDqII}
\end{eqnarray}
Ultimately we want to combine the summations in \refeq{eq:plrcDq} and
\refeq{eq:plrcDqII} and thus the limits of the summations must be the
same.  The limits of the summation in \refeq{eq:plrcDqII} can be
adjusting by using a new index $i'=i-1$ (thus $i=i'+1$).  Substituting
$i'$ for $i$, \refeq{eq:plrcDqII} can be written
\begin{eqnarray}
  \Dvec^{q+1} &=& 
    \Evec^{q+1} 
     \epsilon_0(\epsilon_\infty + \chi^0 - \xi^0)
     + \Evec^q \epsilon_0 \xi^0 \nonumber\\
  && +
    \epsilon_0\sum_{i'=0}^{q-1}\left[
    \Evec^{q-i'} \chi^{i'+1}
    + \left(\Evec^{q-i'-1}-\Evec^{q-i'}\right)\xi^{i'+1}\right].
  \label{eq:plrcDqIII}
\end{eqnarray}
Since $i'$ is just an index, we can return to calling is merely $i$.

The temporal finite-difference of the flux density is obtained by
combining \refeq{eq:plrcDq} and \refeq{eq:plrcDqIII}.  The result is
\begin{eqnarray}
  \frac{\Dvec^{q+1} - \Dvec^{q}}{\Delt} &=& 
  \frac{1}{\Delt}\biggl(
    \Evec^{q+1} 
     \epsilon_0(\epsilon_\infty - \chi^0 + \xi^0)
     + \Evec^q \epsilon_0(-\epsilon_\infty + \xi^0) \nonumber\\
  &&\qquad-
     \epsilon_0\sum_{i=0}^{q-1}\left[
    \Evec^{q-i} \Delta\chi^{i}
    + \left(\Evec^{q-i-1}-\Evec^{q-i}\right)\Delta\xi^{i}\right]
	\biggr)
   \label{eq:plrcDdifference}
\end{eqnarray}
where
\begin{eqnarray}
  \Delta\chi^i &=& \chi^i -  \chi^{i+1}, \\
  \Delta\xi^i &=& \xi^i -  \xi^{i+1}.
\end{eqnarray}

The summation in \refeq{eq:plrcDdifference} does not contain
$\Evec^{q+1}$.  Hence, using \refeq{eq:plrcDdifference} to replace the
right side of \refeq{eq:plrcAmpere} and solving for $\Evec^{q+1}$
yields
\begin{equation}
  \Evec^{q+1}  = 
    \frac{\epsilon_\infty - \xi^0}{\epsilon_\infty + \chi^0 - \xi^0}
    \Evec^q
    +
    \frac{\frac{\Delt}{\epsilon_0}}{\epsilon_\infty + \chi^0 - \xi^0} 
    \nabla\times\Hvec^{q+1/2} + 
    \frac{1}{\epsilon_\infty + \chi^0 - \xi^0} \Psivec^q
  \label{eq:plrcEUpdate}
\end{equation}
where $\Psivec^q$, known as the recursive accumulator, is given by
\begin{equation}
  \Psivec^q = \sum_{i=0}^{q-1}\left[
    \Evec^{q-i} \Delta\chi^{i}
    + \left(\Evec^{q-i-1}-\Evec^{q-i}\right)\Delta\xi^{i}\right]
  \label{eq:plrcPsiDef}
\end{equation}

Equation \refeq{eq:plrcEUpdate} is used to update the electric field.
It appears that a summation must be computed which requires knowledge
of all the previous values of the electric field.  Clearly this would
be prohibitive if this were the case in practice.  Fortunately,
provided the material impulse response can be expressed in terms of
exponentials, there is a recursive formulation which can be used to
efficiently express this summation.

Consider $\Psivec^q$ with the $i=0$ term written explicitly, i.e., 
\begin{equation}
  \Psivec^q = \Evec^{q} (\Delta\chi^{0} - \Delta\xi^{0})
    + \Evec^{q-1}\Delta\xi^{0} +
  \sum_{i=1}^{q-1}\bigl[
    \Evec^{q-i} \Delta\chi^{i}
    + \left(\Evec^{q-i-1}-\Evec^{q-i}\right)\Delta\xi^{i}\bigr].
\end{equation}
Employing a change of indices for the summation so that the new
limits range from $0$ to $q-2$, this can be written:
\begin{equation}
  \Psivec^q = \Evec^{q} (\Delta\chi^{0} - \Delta\xi^{0})
    + \Evec^{q-1}\Delta\xi^{0} +
  \sum_{i=0}^{q-2}\bigl[
    \Evec^{q-i-1} \Delta\chi^{i+1}
    + \left(\Evec^{q-i-2}-\Evec^{q-i-1}\right)\Delta\xi^{i+1}\bigr].
  \label{eq:plrcPsiDefI}
\end{equation}
Now consider $\Psivec^{q-1}$ by writing \refeq{eq:plrcPsiDef} with $q$
replaced by $q-1$:
\begin{equation}
  \Psivec^{q-1} = \sum_{i=0}^{q-2}\left[
    \Evec^{q-i-1} \Delta\chi^{i}
    + \left(\Evec^{q-i-2}-\Evec^{q-i-1}\right)\Delta\xi^{i}\right]
  \label{eq:plrcPsiQminusOne}
\end{equation}
Note the similarity between the summation in \refeq{eq:plrcPsiDefI}
and the right-hand side of \refeq{eq:plrcPsiQminusOne}.  These are the
same except in \refeq{eq:plrcPsiDefI} the summation involves
$\Delta\chi^{i+1}$ and $\Delta\xi^{i+1}$ while in
\refeq{eq:plrcPsiQminusOne} the summation involves $\Delta\chi^i$
and $\Delta\xi^i$.  As we will see, for certain materials it is
possible to relate these values to each other in a simple way.
Specifically, we will find that these are related by
\begin{eqnarray}
  \Delta\chi^{i+1} &=& \crec \Delta\chi^{i}, 
  \label{eq:CrecDefChi} \\
  \Delta\xi^{i+1} &=& \crec \Delta\xi^{i},
  \label{eq:CrecDefXi}
\end{eqnarray}
where $\crec$ is a ``recursion constant'' (which is yet to be
determined).  Given that this recursion relationship exists for 
$\Delta\chi^{i+1}$ and $\Delta\xi^{i+1}$, 
\refeq{eq:plrcPsiDefI} can be written
\begin{eqnarray}
  \Psivec^q &=& \Evec^{q} (\Delta\chi^{0} - \Delta\xi^{0})
    + \Evec^{q-1}\Delta\xi^{0} +
  \crec \sum_{i=0}^{q-2}\bigl[
    \Evec^{q-i-1} \Delta\chi^{i}
    + \left(\Evec^{q-i-2}-\Evec^{q-i-1}\right)\Delta\xi^{i}\bigr], 
  \nonumber\\
  &=& \Evec^{q} (\Delta\chi^{0} - \Delta\xi^{0})
    + \Evec^{q-1}\Delta\xi^{0} +
  \crec \Psivec^{q-1},
\end{eqnarray}
or, after replacing $q$ with $q+1$, this becomes
\begin{equation}
  \Psivec^{q+1} = \Evec^{q+1} (\Delta\chi^{0} - \Delta\xi^{0})
    + \Evec^{q}\Delta\xi^{0} + \crec \Psivec^q.
    \label{eq:plrcPsiDefII}
\end{equation}

The PLRC algorithm is now, at least in the abstract sense, complete.
The implementation is as follows:
\begin{enumerate}
\item Update the magnetic field in the usual way (perhaps using a
dispersive formulation).  
\item Update the electric field using \refeq{eq:plrcEUpdate} (being
sure to first store the previous value of the electric field).
\item Updated the recursive accumulator as specified by
\refeq{eq:plrcPsiDefII} (using both the updated electric field and the
stored value).
\item Repeat.
\end{enumerate}
It now remains to determine the various constants for a given
material.  Specifically, one must know $\chi^0$, $\xi^0$,
$\epsilon_\infty$ (which appear in \refeq{eq:plrcEUpdate}), as well as
$\Delta\chi^0$, $\Delta\xi^0$, and $\crec$ (which appear in
\refeq{eq:plrcPsiDefII}).

\section{PLRC for Debye Material}

The time-domain form of the susceptibility function for Deybe
materials was given in \refeq{eq:debyeChiTime}.  Using this in
\refeq{eq:ChiIDef} and \refeq{eq:XiIDef} yields
\begin{eqnarray}
  \chi^i &=& \epsilon_d\left(1-e^{-\Delt/\tau_d}\right)e^{-i\Delt/\tau_d},
  \label{eq:plrcChiDebye} \\
  \xi^i &=& \frac{\epsilon_d\tau_d}{\Delt}
            \left(1-\left[\frac{\Delt}{\tau_d}+1\right]
                  e^{-\Delt/\tau_d}\right)e^{-i\Delt/\tau_d}.
  \label{eq:plrcXiDebye}
\end{eqnarray}
Setting $i$ equal to zero yields
\begin{eqnarray}
  \chi^0 &=& \epsilon_d\left(1-e^{-\Delt/\tau_d}\right), \\
  \xi^0 &=& \frac{\epsilon_d\tau_d}{\Delt}
            \left(1-\left[\frac{\Delt}{\tau_d}+1\right]
                  e^{-\Delt/\tau_d}\right).
\end{eqnarray}
The time-constant $\tau_d$ can be expressed in terms of multiples of
the time step $\Delt$, e.g., $\tau_d=N_d\Delt$.  Note that the time
step is 
always divided by $\tau_d$ in these expressions so that the only important
consideration is the ratio of these quantities, i.e., $\Delt/\tau_d = 1/N_d$.

From \refeq{eq:plrcChiDebye} we observe
\begin{eqnarray}
  \chi^{i+1} &=&
                 \epsilon_d\left(1-e^{-\Delt/\tau_d}\right)
                 e^{-(i+1)\Delt/\tau_d}, \nonumber\\
             &=& e^{-\Delt/\tau_d}
                 \epsilon_d\left(1-e^{-\Delt/\tau_d}\right)
                 e^{-i\Delt/\tau_d},\nonumber\\
             &=& e^{-\Delt/\tau_d} \chi^i.
\end{eqnarray}
Now consider $\Delta\chi^i$ which is given by
\begin{eqnarray}
  \Delta\chi^i &=& \chi^i - \chi^{i+1}, \nonumber\\
              &=& \chi^i - e^{-\Delt/\tau_d}\chi^i, \nonumber\\
              &=& \chi^i(1 - e^{-\Delt/\tau_d}).
\end{eqnarray}
Thus $\Delta\chi^{i+1}$ can be written as
\begin{eqnarray}
  \Delta\chi^{i+1} &=& \chi^{i+1}(1 - e^{-\Delt/\tau_d}), \nonumber\\
        &=& e^{-\Delt/\tau_d}\chi^{i}(1 - e^{-\Delt/\tau_d}), \nonumber\\
        &=& e^{-\Delt/\tau_d}\Delta\chi^{i}.
            \label{eq:prlcDelChiRecursion}
\end{eqnarray}
Similar arguments pertain to $\xi^i$ and $\Delta\xi^i$ resulting in
\begin{equation}
  \Delta\xi^{i+1} =e^{-\Delt/\tau_d} \Delta\xi^i.
            \label{eq:prlcDelXiRecursion}
\end{equation}

From \refeq{eq:prlcDelChiRecursion} and \refeq{eq:prlcDelXiRecursion},
and in accordance with the description of $\crec$ given in
\refeq{eq:CrecDefChi} and \refeq{eq:CrecDefXi}, we conclude that
\begin{equation}
  \crec = e^{-\Delt/\tau_d}.
\end{equation}
The factor $\exp(-\Delt/\tau_d)$, i.e., $\crec$, appears in several of
the terms given above.  Writing the ratio $\Delt/\tau_d$ as $1/N_d$,
all the terms involved in the implementation of the PLRC method can be
expressed as
\begin{eqnarray}
  \crec &=& e^{-1/N_d}, \\
  \chi^0 &=& \epsilon_d\left(1-\crec\right), \\
  \xi^0  &=& \epsilon_d N_d
             \left(1-\left[\frac{1}{N_d}+1\right]\crec\right), \\
  \Delta\chi^0 &=& \chi^0(1-\crec), \\
  \Delta\xi^0  &=& \xi^0(1-\crec).
\end{eqnarray}

Nearly all the terms involved in the PLRC method are unitless and
independent of scale.  The only term which is not is
$\Delt/\epsilon_0$ which multiplies the curl of the magnetic field in
\refeq{eq:plrcEUpdate}.  However, this is a term which has appeared in
all the electric-field update equations we have ever considered and,
after extracting the $1/\delta$ inherent in the finite-difference form
of the curl operator, it can be expressed in terms of the Courant
number and characteristic impedance, i.e.,
$\Delt/(\delta\epsilon_0)=S_c\eta_0$.
